% Options for packages loaded elsewhere
\PassOptionsToPackage{unicode}{hyperref}
\PassOptionsToPackage{hyphens}{url}
%
\documentclass[
]{article}
\usepackage{amsmath,amssymb}
\usepackage{lmodern}
\usepackage{iftex}
\ifPDFTeX
  \usepackage[T1]{fontenc}
  \usepackage[utf8]{inputenc}
  \usepackage{textcomp} % provide euro and other symbols
\else % if luatex or xetex
  \usepackage{unicode-math}
  \defaultfontfeatures{Scale=MatchLowercase}
  \defaultfontfeatures[\rmfamily]{Ligatures=TeX,Scale=1}
\fi
% Use upquote if available, for straight quotes in verbatim environments
\IfFileExists{upquote.sty}{\usepackage{upquote}}{}
\IfFileExists{microtype.sty}{% use microtype if available
  \usepackage[]{microtype}
  \UseMicrotypeSet[protrusion]{basicmath} % disable protrusion for tt fonts
}{}
\makeatletter
\@ifundefined{KOMAClassName}{% if non-KOMA class
  \IfFileExists{parskip.sty}{%
    \usepackage{parskip}
  }{% else
    \setlength{\parindent}{0pt}
    \setlength{\parskip}{6pt plus 2pt minus 1pt}}
}{% if KOMA class
  \KOMAoptions{parskip=half}}
\makeatother
\usepackage{xcolor}
\usepackage[margin=1in]{geometry}
\usepackage{color}
\usepackage{fancyvrb}
\newcommand{\VerbBar}{|}
\newcommand{\VERB}{\Verb[commandchars=\\\{\}]}
\DefineVerbatimEnvironment{Highlighting}{Verbatim}{commandchars=\\\{\}}
% Add ',fontsize=\small' for more characters per line
\usepackage{framed}
\definecolor{shadecolor}{RGB}{248,248,248}
\newenvironment{Shaded}{\begin{snugshade}}{\end{snugshade}}
\newcommand{\AlertTok}[1]{\textcolor[rgb]{0.94,0.16,0.16}{#1}}
\newcommand{\AnnotationTok}[1]{\textcolor[rgb]{0.56,0.35,0.01}{\textbf{\textit{#1}}}}
\newcommand{\AttributeTok}[1]{\textcolor[rgb]{0.77,0.63,0.00}{#1}}
\newcommand{\BaseNTok}[1]{\textcolor[rgb]{0.00,0.00,0.81}{#1}}
\newcommand{\BuiltInTok}[1]{#1}
\newcommand{\CharTok}[1]{\textcolor[rgb]{0.31,0.60,0.02}{#1}}
\newcommand{\CommentTok}[1]{\textcolor[rgb]{0.56,0.35,0.01}{\textit{#1}}}
\newcommand{\CommentVarTok}[1]{\textcolor[rgb]{0.56,0.35,0.01}{\textbf{\textit{#1}}}}
\newcommand{\ConstantTok}[1]{\textcolor[rgb]{0.00,0.00,0.00}{#1}}
\newcommand{\ControlFlowTok}[1]{\textcolor[rgb]{0.13,0.29,0.53}{\textbf{#1}}}
\newcommand{\DataTypeTok}[1]{\textcolor[rgb]{0.13,0.29,0.53}{#1}}
\newcommand{\DecValTok}[1]{\textcolor[rgb]{0.00,0.00,0.81}{#1}}
\newcommand{\DocumentationTok}[1]{\textcolor[rgb]{0.56,0.35,0.01}{\textbf{\textit{#1}}}}
\newcommand{\ErrorTok}[1]{\textcolor[rgb]{0.64,0.00,0.00}{\textbf{#1}}}
\newcommand{\ExtensionTok}[1]{#1}
\newcommand{\FloatTok}[1]{\textcolor[rgb]{0.00,0.00,0.81}{#1}}
\newcommand{\FunctionTok}[1]{\textcolor[rgb]{0.00,0.00,0.00}{#1}}
\newcommand{\ImportTok}[1]{#1}
\newcommand{\InformationTok}[1]{\textcolor[rgb]{0.56,0.35,0.01}{\textbf{\textit{#1}}}}
\newcommand{\KeywordTok}[1]{\textcolor[rgb]{0.13,0.29,0.53}{\textbf{#1}}}
\newcommand{\NormalTok}[1]{#1}
\newcommand{\OperatorTok}[1]{\textcolor[rgb]{0.81,0.36,0.00}{\textbf{#1}}}
\newcommand{\OtherTok}[1]{\textcolor[rgb]{0.56,0.35,0.01}{#1}}
\newcommand{\PreprocessorTok}[1]{\textcolor[rgb]{0.56,0.35,0.01}{\textit{#1}}}
\newcommand{\RegionMarkerTok}[1]{#1}
\newcommand{\SpecialCharTok}[1]{\textcolor[rgb]{0.00,0.00,0.00}{#1}}
\newcommand{\SpecialStringTok}[1]{\textcolor[rgb]{0.31,0.60,0.02}{#1}}
\newcommand{\StringTok}[1]{\textcolor[rgb]{0.31,0.60,0.02}{#1}}
\newcommand{\VariableTok}[1]{\textcolor[rgb]{0.00,0.00,0.00}{#1}}
\newcommand{\VerbatimStringTok}[1]{\textcolor[rgb]{0.31,0.60,0.02}{#1}}
\newcommand{\WarningTok}[1]{\textcolor[rgb]{0.56,0.35,0.01}{\textbf{\textit{#1}}}}
\usepackage{longtable,booktabs,array}
\usepackage{calc} % for calculating minipage widths
% Correct order of tables after \paragraph or \subparagraph
\usepackage{etoolbox}
\makeatletter
\patchcmd\longtable{\par}{\if@noskipsec\mbox{}\fi\par}{}{}
\makeatother
% Allow footnotes in longtable head/foot
\IfFileExists{footnotehyper.sty}{\usepackage{footnotehyper}}{\usepackage{footnote}}
\makesavenoteenv{longtable}
\usepackage{graphicx}
\makeatletter
\def\maxwidth{\ifdim\Gin@nat@width>\linewidth\linewidth\else\Gin@nat@width\fi}
\def\maxheight{\ifdim\Gin@nat@height>\textheight\textheight\else\Gin@nat@height\fi}
\makeatother
% Scale images if necessary, so that they will not overflow the page
% margins by default, and it is still possible to overwrite the defaults
% using explicit options in \includegraphics[width, height, ...]{}
\setkeys{Gin}{width=\maxwidth,height=\maxheight,keepaspectratio}
% Set default figure placement to htbp
\makeatletter
\def\fps@figure{htbp}
\makeatother
\setlength{\emergencystretch}{3em} % prevent overfull lines
\providecommand{\tightlist}{%
  \setlength{\itemsep}{0pt}\setlength{\parskip}{0pt}}
\setcounter{secnumdepth}{-\maxdimen} % remove section numbering
\usepackage{tikz}
\usepackage{fvextra}
\DefineVerbatimEnvironment{Highlighting}{Verbatim}{breaklines,commandchars=\\\{\}}
\ifLuaTeX
  \usepackage{selnolig}  % disable illegal ligatures
\fi
\IfFileExists{bookmark.sty}{\usepackage{bookmark}}{\usepackage{hyperref}}
\IfFileExists{xurl.sty}{\usepackage{xurl}}{} % add URL line breaks if available
\urlstyle{same} % disable monospaced font for URLs
\hypersetup{
  pdftitle={Using ABC to Infer Parameters of a Simulated Zombie Epidemic},
  pdfauthor={Henry Bourne, Emma Tarmey, Rachel Wood},
  hidelinks,
  pdfcreator={LaTeX via pandoc}}

\title{Using ABC to Infer Parameters of a Simulated Zombie Epidemic}
\author{Henry Bourne, Emma Tarmey, Rachel Wood}
\date{June 2023}

\begin{document}
\maketitle

\hypertarget{the-revisit-package}{%
\section{The REVISIT package}\label{the-revisit-package}}

\textbf{Need to create a package with appropriate folders and documentation - am happy to start on this}

\hypertarget{documentation}{%
\subsection{Documentation}\label{documentation}}

\hypertarget{integration-of-r-and-c}{%
\subsection{Integration of R and C++}\label{integration-of-r-and-c}}

\hypertarget{simulations}{%
\subsection{Simulations}\label{simulations}}

\begin{Shaded}
\begin{Highlighting}[]
\CommentTok{\# {-}{-}{-}{-}{-} Working Examples {-}{-}{-}{-}{-}}

\CommentTok{\# Replace source call with install.github() and library calls when package is finished!}
\CommentTok{\# install.github("https://github.com/RaspberryEmma/Intractable{-}Models{-}Sim{-}Study")}
\CommentTok{\# library(intractmodelinf)}

\FunctionTok{source}\NormalTok{(}\StringTok{"intractmodelinf/R/generate\_data.R"}\NormalTok{)}
\FunctionTok{suppressPackageStartupMessages}\NormalTok{(\{}
  \FunctionTok{library}\NormalTok{(dplyr)}
  \FunctionTok{library}\NormalTok{(ggplot2)}
  \FunctionTok{library}\NormalTok{(knitr)}
  \FunctionTok{library}\NormalTok{(tidyverse)}
\NormalTok{\})}

\NormalTok{bristol.uni.N }\OtherTok{\textless{}{-}} \DecValTok{29434}
\NormalTok{bristol.N     }\OtherTok{\textless{}{-}} \DecValTok{467099}
\NormalTok{UK.N          }\OtherTok{\textless{}{-}} \DecValTok{67081234}

\NormalTok{bristol.uni.example }\OtherTok{\textless{}{-}} \FunctionTok{generate.SIR.data}\NormalTok{(}\AttributeTok{total.N     =}\NormalTok{ bristol.uni.N,}
                                         \AttributeTok{initial.inf =} \DecValTok{10}\NormalTok{,}
                                         \AttributeTok{total.T     =} \DecValTok{150}\NormalTok{)}

\NormalTok{bristol.example }\OtherTok{\textless{}{-}} \FunctionTok{generate.SIR.data}\NormalTok{(}\AttributeTok{total.N     =}\NormalTok{ bristol.N,}
                                     \AttributeTok{initial.inf =} \DecValTok{10}\NormalTok{,}
                                     \AttributeTok{total.T     =} \DecValTok{150}\NormalTok{)}

\NormalTok{UK.example }\OtherTok{\textless{}{-}} \FunctionTok{generate.SIR.data}\NormalTok{(}\AttributeTok{total.N     =}\NormalTok{ UK.N,}
                                \AttributeTok{initial.inf =} \DecValTok{10}\NormalTok{,}
                                \AttributeTok{total.T     =} \DecValTok{150}\NormalTok{)}

\NormalTok{bristol.uni.example }\SpecialCharTok{\%\textgreater{}\%} \FunctionTok{head}\NormalTok{() }\SpecialCharTok{\%\textgreater{}\%}\NormalTok{ knitr}\SpecialCharTok{::}\FunctionTok{kable}\NormalTok{()}
\end{Highlighting}
\end{Shaded}

\begin{longtable}[]{@{}rrrrrrr@{}}
\toprule()
t & S.t & I.t & R.t & N & b & k \\
\midrule()
\endhead
1 & 29434 & 10 & 0 & 29434 & 0.6323166 & 0.3918715 \\
2 & 29427 & 12 & 3 & 29434 & 0.6323166 & 0.3918715 \\
3 & 29419 & 14 & 7 & 29434 & 0.6323166 & 0.3918715 \\
4 & 29410 & 17 & 12 & 29434 & 0.6323166 & 0.3918715 \\
5 & 29399 & 21 & 18 & 29434 & 0.6323166 & 0.3918715 \\
6 & 29385 & 26 & 26 & 29434 & 0.6323166 & 0.3918715 \\
\bottomrule()
\end{longtable}

\begin{Shaded}
\begin{Highlighting}[]
\CommentTok{\# {-}{-}{-}{-}{-} Save Data {-}{-}{-}{-}{-}}

\FunctionTok{write.csv}\NormalTok{( bristol.uni.example, }\AttributeTok{file =} \StringTok{"data/bristol\_uni\_example.csv"}\NormalTok{, }\AttributeTok{row.names =} \ConstantTok{FALSE}\NormalTok{ )}
\FunctionTok{write.csv}\NormalTok{( bristol.example,     }\AttributeTok{file =} \StringTok{"data/bristol\_example.csv"}\NormalTok{, }\AttributeTok{row.names =} \ConstantTok{FALSE}\NormalTok{ )}
\FunctionTok{write.csv}\NormalTok{( UK.example,          }\AttributeTok{file =} \StringTok{"data/UK\_example.csv"}\NormalTok{, }\AttributeTok{row.names =} \ConstantTok{FALSE}\NormalTok{ )}
\end{Highlighting}
\end{Shaded}

\hypertarget{parallelisation}{%
\subsection{Parallelisation}\label{parallelisation}}

\hypertarget{generating-szr-data}{%
\section{Generating SZR data}\label{generating-szr-data}}

We simulate the zombie epidemic using the \texttt{generate.SIR.data()}
function, which is given by the code below. For ease, this function has
also been included in our \texttt{intractmodelinf} package

\begin{Shaded}
\begin{Highlighting}[]
\NormalTok{generate.SIR.data }\OtherTok{\textless{}{-}} \ControlFlowTok{function}\NormalTok{(}\AttributeTok{total.N =} \ConstantTok{NULL}\NormalTok{, }\AttributeTok{initial.inf =} \ConstantTok{NULL}\NormalTok{, }\AttributeTok{total.T =} \ConstantTok{NULL}\NormalTok{) \{}
  
  \CommentTok{\# initial model conditions}
\NormalTok{  cond }\OtherTok{\textless{}{-}} \FunctionTok{generate.start.cond}\NormalTok{(}\AttributeTok{total.N =}\NormalTok{ total.N, }\AttributeTok{initial.inf =}\NormalTok{ initial.inf)}
  
  \CommentTok{\# constant model parameters}
\NormalTok{  N    }\OtherTok{\textless{}{-}}\NormalTok{ total.N}
\NormalTok{  b    }\OtherTok{\textless{}{-}}\NormalTok{ cond[}\DecValTok{5}\NormalTok{]}
\NormalTok{  k    }\OtherTok{\textless{}{-}}\NormalTok{ cond[}\DecValTok{6}\NormalTok{]}
  
  \CommentTok{\# results array to hold values of each function at times 1 to total.T}
\NormalTok{  results }\OtherTok{\textless{}{-}} \FunctionTok{array}\NormalTok{(}\AttributeTok{data =} \ConstantTok{NA}\NormalTok{, }\AttributeTok{dim =} \FunctionTok{c}\NormalTok{(total.T, }\FunctionTok{length}\NormalTok{(cond)}\SpecialCharTok{+}\DecValTok{1}\NormalTok{) )}
\NormalTok{  results[}\DecValTok{1}\NormalTok{, ] }\OtherTok{\textless{}{-}} \FunctionTok{c}\NormalTok{(}\DecValTok{1}\NormalTok{, cond)}
  
  \CommentTok{\# vector to hold current sim values}
\NormalTok{  values.t }\OtherTok{\textless{}{-}} \ConstantTok{NULL}
  
  \CommentTok{\# run through SIR simulation for times 2 to total.T}
  \CommentTok{\# results vector indices 1=S, 2=I, 3=R}
  \ControlFlowTok{for}\NormalTok{ (t }\ControlFlowTok{in} \DecValTok{2}\SpecialCharTok{:}\NormalTok{total.T) \{}
\NormalTok{    S.t }\OtherTok{\textless{}{-}} \FunctionTok{as.integer}\NormalTok{( results[t}\DecValTok{{-}1}\NormalTok{, }\DecValTok{2}\NormalTok{] }\SpecialCharTok{+}\NormalTok{ N}\SpecialCharTok{*}\FunctionTok{change.s}\NormalTok{(b,    results[t}\DecValTok{{-}1}\NormalTok{, }\DecValTok{2}\NormalTok{]}\SpecialCharTok{/}\NormalTok{N, results[t}\DecValTok{{-}1}\NormalTok{, }\DecValTok{3}\NormalTok{]}\SpecialCharTok{/}\NormalTok{N) )}
    
\NormalTok{    I.t }\OtherTok{\textless{}{-}} \FunctionTok{as.integer}\NormalTok{( results[t}\DecValTok{{-}1}\NormalTok{, }\DecValTok{3}\NormalTok{] }\SpecialCharTok{+}\NormalTok{ N}\SpecialCharTok{*}\FunctionTok{change.i}\NormalTok{(b, k, results[t}\DecValTok{{-}1}\NormalTok{, }\DecValTok{2}\NormalTok{]}\SpecialCharTok{/}\NormalTok{N, results[t}\DecValTok{{-}1}\NormalTok{, }\DecValTok{3}\NormalTok{]}\SpecialCharTok{/}\NormalTok{N) )}
    
\NormalTok{    R.t }\OtherTok{\textless{}{-}} \FunctionTok{as.integer}\NormalTok{( results[t}\DecValTok{{-}1}\NormalTok{, }\DecValTok{4}\NormalTok{] }\SpecialCharTok{+}\NormalTok{ N}\SpecialCharTok{*}\FunctionTok{change.r}\NormalTok{(k,    results[t}\DecValTok{{-}1}\NormalTok{, }\DecValTok{3}\NormalTok{]}\SpecialCharTok{/}\NormalTok{N) )}
    
\NormalTok{    values.t     }\OtherTok{\textless{}{-}} \FunctionTok{c}\NormalTok{(t, S.t, I.t, R.t, N, b, k)}
\NormalTok{    results[t, ] }\OtherTok{\textless{}{-}}\NormalTok{ values.t}
\NormalTok{  \}}
  
  \CommentTok{\# convert to data.frame to explicitly record column meaning}
\NormalTok{  results           }\OtherTok{\textless{}{-}} \FunctionTok{as.data.frame}\NormalTok{(results)}
  \FunctionTok{colnames}\NormalTok{(results) }\OtherTok{\textless{}{-}} \FunctionTok{c}\NormalTok{(}\StringTok{"t"}\NormalTok{, }\StringTok{"S.t"}\NormalTok{, }\StringTok{"I.t"}\NormalTok{, }\StringTok{"R.t"}\NormalTok{, }\StringTok{"N"}\NormalTok{, }\StringTok{"b"}\NormalTok{, }\StringTok{"k"}\NormalTok{)}
  
  \FunctionTok{return}\NormalTok{ ( results )}
\NormalTok{\}}
\end{Highlighting}
\end{Shaded}

We use this to simulate an outbreak for 3 diff

\hypertarget{implementing-abc}{%
\section{Implementing ABC}\label{implementing-abc}}

\hypertarget{checking-results}{%
\section{Checking Results}\label{checking-results}}

\begin{Shaded}
\begin{Highlighting}[]
\NormalTok{bristol.uni.example }\OtherTok{\textless{}{-}} \FunctionTok{read.csv}\NormalTok{(}\StringTok{"data/bristol\_uni\_example.csv"}\NormalTok{, }\AttributeTok{header =} \ConstantTok{TRUE}\NormalTok{)}
\NormalTok{bristol.example     }\OtherTok{\textless{}{-}} \FunctionTok{read.csv}\NormalTok{(}\StringTok{"data/bristol\_example.csv"}\NormalTok{, }\AttributeTok{header =} \ConstantTok{TRUE}\NormalTok{)}
\NormalTok{UK.example          }\OtherTok{\textless{}{-}} \FunctionTok{read.csv}\NormalTok{(}\StringTok{"data/UK\_example.csv"}\NormalTok{, }\AttributeTok{header =} \ConstantTok{TRUE}\NormalTok{)}

\NormalTok{legend.colors }\OtherTok{\textless{}{-}} \FunctionTok{c}\NormalTok{(}\StringTok{"Susceptible"} \OtherTok{=} \StringTok{"red"}\NormalTok{, }\StringTok{"Infected"} \OtherTok{=} \StringTok{"blue"}\NormalTok{, }\StringTok{"Recovered"} \OtherTok{=} \StringTok{"green"}\NormalTok{)}

\FunctionTok{plot.SIR}\NormalTok{(bristol.uni.example)}
\end{Highlighting}
\end{Shaded}


\end{document}
